\chapter{Introduction}
\label{intro}
%What will happen if we place a team of robots into a soccer field? It is obvious for everyone to realize that nothing is going to happen. This occurs due to the fact that machines, such as robots, should be programmed to perceive their surroundings and act just like human soccer players. Therefore, everything in the robots' world start from scratch. Even if these robots had a perfect sense of their environment, it would be difficult for them to start taking part into the game immediately. There are plenty of things that have to be done before these robots start playing in the way human players do.


Any team participating in a team sport requires both individual and team skills in order to be successful. For human teams, these skills are inherent and naturally improve over time. However, for robot teams these skills must be programmed by the designers of the team. Robotic soccer, known as RoboCup, represents a complex, stochastic, real-time, multi-agent, competitive domain. In such domains, team skills are as important as individual player skills, considering that in soccer simulation leagues there are up to 9 or 11 players per team.
 
%A simulation soccer game consists of two parts. There is a server which has the responsibility of sending perception messages to the agents, as well as, receiving effector messages from the agents to apply them into the soccer field. The second part is the agents which are processes running independently from each other without being able to communicate directly but only with the server.
%In the beginning there must be a connection with the simulation server. When we ensure that we are connected with the server, we are ready to proceed to the next steps.
%Server sends to each connected agent messages every 20ms, these messages include information about agent's vision and other perceptions. Each agent parses these messages to update his perceptions, At the end of the parsing the agent knows the values of every joint of his body,  he has also knowledge about the location in relation to his body of every landmark, the ball and other players which are in the field of his view and finally possible messages from teammates. Now, agent is ready to continue to the main procedure of thinking.
%First of all, agent has to calculate his position in the soccer field, it is not so simple as it sounds and it requires at least two landmarks in the field of our view. We are going to explain this operation extensively later.
%Even if, our agent knows his positions in the soccer field and is able to calculate the position of every other agent in his sight, as well as, the soccer ball position, he is still not able to perform a single action. 
%This will be feasible if he combines motions which are going to help him perform each action. 
%Even in real life, a human soccer player has to combine simple movements for example, walking, turning and kicking, to perform a kick towards the opponents' goal. 
%The same principle applies in simulation soccer too. In our approach, we have categorize the actions in relation to their complexity. At first simple actions, which just use motions in order to be completed. 
%We continue with more complex actions which make use of more than one simple actions to be executed by the agent with success. 
%An example of a simple action is a turn towards the ball and a more complex action could be walking to a specific coordinate in the soccer field. We can realize that a complex action such as the above is going to make use of more than one simple actions and movements. 
%Until now, we have accomplished every agent in the field to be able to recognize objects, find its position and do simple and complex actions.
%Returning to the first question which we have put in the beginning of this introduction, we could answer with certainty that every agent in the soccer field now has a complete sense of its surroundings and is able to perform actions which are able to make changes in his environment.
%Even so, these improvements are not going to bring success to the team, agents have not the ability to communicate with their team-mates and reasonably they are not able to coordinate their actions. Even humans since the advent of their history form all kinds of groups striving to achieve a common goal, especially , for teams participating in games, where success can only be achieved through collaborative and coordinated efforts. As we realize, coordination and cooperation are the last pieces of the puzzle. This two team skills are going to be accomplished through communication process. 
%This thesis as well as a proposed solution of all the problems generated in robotic soccer. 
%The main objective is to develop an efficient software system to correctly model the behaviors of simulated Nao robots in such a competitive environment as the simulation soccer league. Additionally, we are coming up with an approach in which agents coordinate through the communication channel their actions  which will be calculated to be costless and worthy for the team.
%The challenging and the most time consuming part of this project was the coordination part which I firmly believe is a skill of major importance either in a simulated team or in a real soccer team.




\section{Thesis Contribution}

This thesis presents a complete team design for the RoboCup 3D Simulation League focusing on player behavior, team strategy, and team coordination. Our agents are designed in a way that enables them to act effectively both autonomously and as members of the team. Initially, the development of the individual player skills is described. These skills include robust self localization and object tracking, effective locomotion and soccer motions, basic and complex action execution, and communication with teammates. Subsequently, a hierarchical coordination protocol is described, which coordinates all the individual player skills yielding a complete behavior for each agent within the frame of a global team strategy. Our approach is based on first sharing and fusing information about the game state and then decomposing the global coordination problem for the 9 or 11 players to smaller coordination problems over dynamically-determined subsets of players adhering to an adaptive global team formation. An exhaustive algorithm is used over the most important subset of active players (the three ones closest to the ball) to derive an optimal set of actions, whereas a less-expensive dynamic programming algorithm is used over the remaining players (support players) to derive their actions. Coordinated actions are evaluated through a function that combines costs related to positions, distances, potential collisions, and field coverage. Our approach and our Java implementation enable the team to compute coordinated actions approximately every two seconds yielding quick responsiveness to dynamically changing game states. The results of complete games against existing teams, some of which compete for several years in the RoboCup 3D Simulation League, reveal that our team is quite competitive mostly thanks to the proposed coordination approach.


\section{Thesis Outline}
Chapter~\ref{background} provides some background information on the RoboCup competition. In Chapter~\ref{Soccer Simulation League 3D} we present the SimSpark simulation platform and the general framework of the challenging RoboCup 3D Simulation League domain we are going to work on. Continuing to Chapter~\ref{Agent}, the core ideas, the architecture of our agents, and the individual player behavior are presented. Moving on to Chapter~\ref{Coordination}, we present in detail our team strategy and our coordination method over a custom communication protocol among the team players. In Chapter~\ref{results} the results and the evaluation of our work are presented though several experiments and test games. Chapter~\ref{related} presents similar systems developed by other RoboCup teams including a brief comparison between those systems and ours. Finally, Chapter~\ref{conclusion} serves as an epilogue to this thesis, including proposals on extending and improving our framework and a discussion about the experience we gained from this work. 
