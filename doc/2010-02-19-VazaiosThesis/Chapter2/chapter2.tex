\chapter{Commmunication}
\label{communication}

\section{Need for Communication}
Communication is the process of transferring information from one entity to another. The need for such a process is obvious as different entities should co-exist, collaborate with each other. Thus , being able to express feelings, desires, needs and information is crucial to all living beings from a tiny cell to a human being. People instinctively gather and form groups with a common goal either important ( i.e. survival ) or not (i.e. soccer team).
The consistency of these groups is proportional to the communication between its members.

\section{Communication in a team }
Communication among team members is a crucial ability for any kind of team striving to achieve a common goal, especially for teams participating in games, where success can only be achieved through collaborative and coordinated efforts. Teams lacking means of communication are doomed to act simply as a collection of individuals with no added benefit, other than the multiplicity of individual skills. For human teams, communication is second nature; it is used not only for teamwork in games, but also in all aspects of life, and takes a multitude of different forms (oral, written, gestural, visual, auditory). 

\section{Communication across robots}
Robots can hardly replicate all human communication means, since these would require extremely accurate and robust perceptual and action abilities on each robot. Fortunately, most modern robots are capable of communicating over data networks, an ability which is not available to their human counterparts. However, exploiting such networking means for team communication purposes in an efficient manner, that does not drain the underlying resources and provides transparent exchange of information in real time is rather challenging problem. Even if data networks are sufficient for exchanging data between robots and computers, there is a lot of research on other fields such as computer vision or sound communication. The research in these areas aim at creating a more human-like robotic platforms , which in the future could be used as a substitute or assistance to humans. This means that except for the network packets robotic systems have to be able to handle other types of messages, like sounds or interpretation of visual images. Unification of all these different messages is easier said than done, however a lot of work has been made in order to have a transparent way for dealing with this diversity.
\section{Communication in Robocup}

A key aspect of most RoboCup leagues is the multi-agent environment; each team features 11 (simulation), 5 (small-size), 6 (mid-size), 3 (humanoid), or 3 (standard platform) robots. These robots cannot simply act as individuals; they must focus on teamwork in order to cope effectively with an unknown opponent team and such teamwork requires communication. Fortunately, the use of a wireless network is allowed in most leagues, however there are strict rules about the available bandwidth for each team with the maximum bit rate varying between 500Kbps and 2.2Mbps.  

A common requirement in almost all RoboCup leagues is that all players must be able to listen to the so-called game controller~\cite{robocupgc}, which is software that acts as a game referee and runs on a remote computer. The game controller broadcasts messages over the wireless network that contain important information about the state of the game (initial positioning period, game start, game time, game completion, goal scored, penalty on a player, etc.). Even
though this is one-way communication (the game controller does not expect any messages from the robots), the ability to receive and decode correctly game controller messages is a crucial communication component for each team. It has even been suggested that teams not listening to and obeying the game controller of the league should be automatically disqualified. 

Each robot of an SPL team typically maintains a local perception obtained through a vision module, a local estimate of self location obtained through a localization module, a local active role which might be dynamically assigned by some behavior module, and a local status which indicates the state of the robot (active, penalized, fallen-down, etc.). To assess the benefits of the ability of transparent intra-team communication, consider the following scenarios: (a) say that robot 3 has spotted the ball, whereas robot 2 is scanning for it; robot 2 could be facilitated in locating the ball through a {\tt perception.robot[3].ball} query to get the (possibly noisy) ball coordinates from robot 3 and actively look for it in that direction; (b) say that robot 2 has lost its estimate of self-location; a {\tt perception.robot[*].teammates} that returns coordinates of visible teammates could greatly facilitate recovery of its own location; (c) say that a robot needs to determine its current role depending on the current formation of the team; a {\tt localization.robot[*].location} query could return the (estimated) locations of all robots in the team and therefore lead to a decision about role assignment; (d) say that the goalie (robot 1) has fallen down leaving the goal unprotected; a {\tt behavior.robot[1].status} could guide a defender to urgently cover the goal in some way. Additionally, the benefits of communication can be valuable to the human developers during debugging: (a) a {\tt
perception.robot[3].camera} query could bring the camera stream of robot 3 to a remote computer; (b) a {\tt
localization.robot[*].location} could bring the (estimated) locations of all robots for display on a remote monitoring
computer; (c) a {\tt behavior.robot[*].status} query could be used to visualize the status of each robot. These examples
indicate that development and actual game play can be greatly facilitated by a transparent and efficient communication
framework. 
% ------------------------------------------------------------------------

%%% Local Variables:
%%% mode: latex
%%% TeX-master: "../thesis"
%%% End:
